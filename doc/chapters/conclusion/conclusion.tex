% !TeX spellcheck = en_US

\chapter{Conclusion and Future Work}\label{chp:conclusion_and_future_work}

\section{Training Data}
A more accurate and balanced data set would probably lead to higher prediction quality. Using images with higher resolution might even enable the neural network to learn the differences between asphalt and gravel roads. Additionally, the predicted masks would probably also be more accurate and fewer wrong predictions would occur. Generally, there might be an increase in the overall quality of predictions.

Additionally, instead of training with a data set consisting of images from a high zoom level (like 18 or 19), using images from a lower zoom level (16 or 17) might result in overall faster prediction as a larger area could be covered at once. Because only the actual changes are of interest, not the predicted contours themselves, it would not be problematic that in this case the prediction accuracy would suffer slightly through a reduced detail lever per image.


\section{QGIS Plugin}
Instead of sending an image and corresponding vectors to the backend, an option could be to send two images – an old and a new one – to the backend. All changes can then be derived from the differences in the predictions of those two images. However, often access to old and updated imagery is lacking, which is a problem for the use case described in the introduction.

Another improvement would be to give the user the option to manually choose a perimeter. The perimeter would then be automatically processed, although it might take a longer time.