% !TeX spellcheck = en_US

\chapter{Overview}\label{chp:overview}
This chapter gives a brief overview over the following chapters and their contents. Firstly, \autoref{chp:state_of_the_art} is about current technologies and developments in the area of computer vision in general and segmentation in particular. \autoref{chp:introduction} is an introduction into the artificial intelligence and how it is helpful and used in this thesis. \autoref{chp:segmentation_with_neural_networks} gives an introduction into convolutional neural networks and their architecture in the context image segmentation. \autoref{chp:practical_challenges} is about practical / technical challenges, that had to be faced during the work. Following, \autoref{chp:theoretical_and_experimental_results} is a collection of theoretical and experimental results, like the participation in the crowdAI mapping challenge or the data generation tool \textit{Airtiler}. \autoref{chp:practical_results} is mainly about the QGIS plugin that has been developed in order to enable end users to use the backend for predictions of their data. Finally, \autoref{chp:conclusion_and_future_work} are some thoughts and ideas, what some next steps may be in a work based on this one.
