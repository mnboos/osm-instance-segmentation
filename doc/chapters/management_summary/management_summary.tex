% !TeX spellcheck = en_US

\chapter*{Management Summary}

\paragraph*{Introduction}
Cadastral surveying means keeping track of real property boundaries. Periodically, the cadastral survey data must be refreshed. Generally, this means finding changes in landcover, among other things, based on aerial imagery. The renewal of the data requires substantial manual work as it is often not trivial to identify the changes.

\paragraph*{Goals}
Using machine learning and QGIS, a modern GIS editing tool, the goal was to reduce the amount of manual work required for renewing the cadastral survey data.

\paragraph*{Results}
The result of this work was a QGIS plugin which supports the user in data renewal. The plugin uses a backend on a remote server, which predicts all objects of an image received from the plugin to find the changes in landcover (deleted, added, or changed) based on the current data. The accuracy of the predictions was about 95\%.

\paragraph*{Future Work}
Future steps would include extending the plugin to allow a large area to be defined, which is then automatically processed in the background. Additionally, by training the neural network with a larger data set, the prediction accuracy could be improved.