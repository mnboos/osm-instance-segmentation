% !TeX spellcheck = en_US

\chapter{State of the art}

There are various different techniques and technologies which allow to build neural networks for machine learning purposes. Due to the increasing computation power of current hardware, especially graphic cards, more complex and therefore computationally heavy models can be built. As a result of this, the active research leads to a number of improvements of existing networks or completely new networks each year.

At the time of this writing, the most promising candidate is Mask R-CNN \cite{He.20170405}, which is based on Faster R-CNN \cite{Ren.20160106}.
Another well performing architecture can be found with U-Net \cite{Ronneberger.20150518b} that has been created for medical purposes. Nonetheless, it can also be used for semantic segmentation of all kind of data.

One big challenge in the area of computer vision is instance segmentation, which not only tries to detect occurences of a specific class, but also tries to find single instances of it. In other words, it tries to find the contour. In order to compare various architectures built for such a purpose, there are different challenges published on the internet. One of those leaderboards is from the Microsoft COCO challenge \cite{Lin.20150221}, which yearly tries to find the most competitive candidates for different tasks, such as object detection, instance segmentation or keypoint detection.