% !TeX spellcheck = en_US

\chapter{Theoretical and Practical Challenges}

\section{Building Outline Regularization}
Once the network has been trained, it can be used to make predictions on images, it has never "seen" before. However, there are situations in which the predictions are far from perfect, especially then if the building is partially covered from trees or has unclear outlines. This can be seen in \autoref{fig:challenges:building_masks}.

\begin{figure}[H]
	\centering
	\begin{subfigure}{0.4\textwidth}
		\centering
    	\includegraphics[width=0.9\linewidth]{chapters/challenges/images/predicted_masks_gt.png}		    \caption{Actual ground truth}
    	\label{fig:challenges:predicted_building_masks_gt}
	\end{subfigure}~
		\begin{subfigure}{0.4\textwidth}
		\centering
    	\includegraphics[width=0.9\linewidth]{chapters/challenges/images/predicted_masks.png}		    \caption{Predicted building masks}
    	\label{fig:challenges:predicted_building_masks}
	\end{subfigure}
	\caption{Predicted masks and actual ground truth}
	\label{fig:challenges:building_masks}
\end{figure}

As a result of the inaccuracies in the predicted building masks the contours of these predictions can not directly be used to create the vectorized outlines. Instead, the predictions have to be regularized. The approach used is similar to \cite{Partovi.2017} and described below.

\begin{enumerate}
	\item Contour extraction
	\item Contour segmentation
	\item Group segments by orientation
\end{enumerate}

\begin{figure}[H]
    \centering
	\begin{subfigure}{0.4\textwidth}
    	\includegraphics[width=0.9\linewidth]{chapters/challenges/images/regular_building_masks.png}		    \caption{Regular buildings}
    	\label{fig:challenges:regular_buildings}
	\end{subfigure}~
	\begin{subfigure}{0.4\textwidth}
    	\includegraphics[width=0.9\linewidth]{chapters/challenges/images/irregular_building_masks.png}       	\caption{Non-orthogonal buildings}
    	\label{fig:challenges:irregular_buildings}
	\end{subfigure}
	\caption{Different kinds of buildings with regard to their corner angles}
	\label{fig:challenges:building_masks}
\end{figure}

The procedure 


\begin{equation}
    x^2 + y^2 = 1
    \label{equ:bsp_chapter:example_equation}
\end{equation}

There are several environments for multi line equations. A very useful one
is \emph{align}, see equation \eqref{equ:bsp_chapter:example_align}.

\begin{align}
    \oint \vec{E} \cdot d \vec{A} &= \frac{q}{\epsilon_0} \\
    \oint \vec{B} \cdot d \vec{A} &= 0 \\
    \oint \vec{E} \cdot d \vec{s} &= - \frac{d \Phi_B}{dt}
    \label{equ:bsp_chapter:example_align}
\end{align}

\marginpar{Figures and Tables}%
Images\index{Image} are always inserted inside a \emph{figure}
environment. If possible, it is advisable to use \texttt{[tb]} as position.
Always remember to add a caption and a label, so you can reference the image
like this: \autoref{fig:bsp_chapter:example_figure}. If possible, images should be
inserted as vector graphics, e.g. \texttt{eps} or \texttt{pdf} - or even drawn
manually in TikZ.

\begin{figure}[t]
    \centering
    \includegraphics[width=2cm]{chapters/bsp_chapter/images/thumbs_up.jpg}
    \caption{An example image}
    \label{fig:bsp_chapter:example_figure}
\end{figure}

Tables\index{Table} can be used quite similarly. They are inserted inside a \emph{table}\index{Table!tabular}
environment, as shown in \autoref{tab:bsp_chapter:example_table}.

\begin{table}[t]
    \centering
    \begin{tabular}{lcrp{4cm}} \toprule
        some & text & is shown & here \\ \midrule
        there is more & text here & and here & cool. \\
        and & even & more & here. \\ \bottomrule
    \end{tabular}
    \caption{A sample table}
    \label{tab:bsp_chapter:example_table}
\end{table}

Another useful tool is the \emph{tabularx} environment. \index{Table!tabularx}
It lets the user specify the total width of the table, instead of each column. 
An example is shown in \autoref{tab:bsp_chapter:example_tabularx}.

\begin{table}[t]
    \centering
    \begin{tabularx}{0.9\linewidth}{lXX} \toprule
        some & text & is shown here \\ \midrule
        there is more & text here & and here. \\
        and & even & more here. \\ \bottomrule
    \end{tabularx}
    \caption{Tabularx example}
    \label{tab:bsp_chapter:example_tabularx}
\end{table}

Please read the documentation of the
\emph{booktabs}\footnote{\url{http://www.ctan.org/pkg/booktabs}} 
package to find information on how to create good tables.
Always remember the first two guidelines and try also to stick to the other three:
\begin{enumerate}
    \item Never, ever use vertical rules.
    \item Never use double rules.
    \item Put the units in the column heading (not in the body of the table).
    \item Always precede a decimal point by a digit; thus $0.1$ \emph{not} just $.1$.
    \item Do not use \enquote{ditto} signs to repeate a value. In many circumstances a blank will serve just as well. If it won't, then repeat the value.
\end{enumerate}

\marginpar{Paragraphs}%
Note that each paragraph is ended with an empty line. 
\emph{Never} use \verb|\\| to end paragraphs -- this is a new line, not a new paragraph.
Also try to keep your source code clean: about 80 characters per line.
Using source control makes your life much easier

\marginpar{Quotes}%
Quotes can easily be made using the \enquote{csquotes} package.
Citing text passages is easily done: \textquote[me][!]{First argument: citation,
second argument: terminal punctuation} Whole block quotes are also easily 
possible.

\blockquote{Formal requirements in academic writing frequently demand that
quotations be embedded in the text if they are short but set oV as a distinct
and typically indented paragraph, a so-called block quotation, if they are
longer than a certain number of lines or words. In the latter case no quotation
marks are inserted.}

\marginpar{SI Units}%
Any numbers and units should by typed using the siunitx package.
Numbers are written as \num{1234} \num{3.45e-5} \numlist{1;2;4} \numrange{4}{30} \ang{10} \ang{5;3;2}.
Units are written with \si{\kilo\gram\meter\per\square\second} or \SI{14}{\farad\tothe{4}}.
Of course also possible is \SIlist{10;40;12}{\meter} or \SIrange{-40}{+125}{\degreeCelsius}
Almost every unit you could possibly think of is implemented!

\marginpar{Bibliography}%
The bibliography is created using \emph{Bibtex}. The
standard format is set to \texttt{ieeetr}, which is the IEEE Standard. There
are example entries for different types of  in the separate bibliography file
\cite{article} \cite{book} \cite{booklet} \cite{conference} \cite{inbook}
\cite{incollection} \cite{manual} \cite{mastersthesis} \cite{misc}
\cite{phdthesis} \cite{proceedings} \cite{techreport} \cite{unpublished}.

\marginpar{Index \& Glossary}%
All glossary entries are made in the separate file \texttt{glossary.tex}.
They can then be used with \gls{equation}.
Acronyms are defined as shown there and used similarly. 
The first time, it will be \emph{\gls{svm}}.
The second time: \emph{\gls{svm}}.
The glossary has to be created manually by invoking \texttt{makeindex -s doku.ist -t doku.glg -o doku.gls doku.glo}.
The index is simply created by using \texttt{index\{text\}}. It is generated automatically.

\subsection{Listings}
Listings are created by the \texttt{lstlistings} package.

\todo{This is a simple ToDo note}

\todonote{This is a small note}
\citationneeded