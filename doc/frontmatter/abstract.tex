\cleardoublepage
\markleft{\abstractname}
\pdfbookmark[1]{Abstract}{abstract}

\chapter*{Abstract}
Cadastral surveying keeps track of real property boundaries, such as buildings, parking lots, and roads. Periodically, every one to five years, the cadastral survey data must be refreshed. Generally, this means finding changes in landcover, based on aerial imagery such as orthophotos, lidar, and terrestrial measurements. The changes consist of new objects that have been added, existing objects that have changed, or objects that have been removed since the last survey. The renewal of these data requires substantial manual work because it is often not easy to identify the changes.

This study uses an approach in which machine learning is applied and a modern geographic information system (GIS) is used as an editing tool. The goal was to reduce the amount of manual labor required. Unfortunately, state-of-the-art technologies are not advanced enough to rely on completely when trying to find these changes. Instead, they can provide support for human work and use artificial intelligence called “intelligence augmentation”. In this study, we analyzed selected objects and classes from landcover in an orthophoto and compared them with the current landcover vector data layer.

The result of this work was a QGIS plugin. The plugin sent an image of the current extent together with loaded cadastral survey data to the backend, which then predicted all objects within the classes using a neural network. The classes were building, highways, vineyard, tennis court and swimming pool. In the next step, these predictions were georeferenced and the actual changes were determined by comparing the old vector data from the cadastral survey to the predictions. Finally, the changes were sent back to QGIS and visualized there according to their type (changed, deleted, added). Because the network was trained using satellite imagery from Microsoft Bing and vector data from OpenStreetMap, only about 80\% to 90\% of the predictions were correct. The errors were due to the satellite imagery passing through a photogrammetric post-processing step, which can lead to displacements regarding OpenStreetMap vector data.
