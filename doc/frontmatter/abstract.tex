\cleardoublepage
\markleft{\abstractname}
\pdfbookmark[1]{Abstract}{abstract}

\chapter*{Abstract}
Periodically, cadastral survey data has to be updated. Generally spoken, this means among others to find changes in landcover based on aerial imagery (orthophotos, lidar, terrestrial measurements), which consists of new objects that have been added, existing objects that have changed or objects that have been removed in contrast to the latest survey data. As a result of this, this data has to be updated periodically, for example all 1-5 years, which results in a lot of manual work, as it is often not trivial to find the changes mentioned above. Due to this, the approach of this work is to apply machine learning and a modern GIS as an editing tool. Unfortunately, the current state of the art technologies are still nod advanced enough, to only rely on it when trying to find these changes. Instead, it is just a support of human work, using artificial intelligence, what is called intelligence augmentation. In this work we concentrate on analyzing selected objects/classes from landcover in an orthophoto, and compare this with the current landcover vector data layer.

The result of this work is a QGIS plugin, which sends an image of the current extent together with loaded cadastral survey data to the backend, which then predicts the all objects within the classes \textit{building}, \textit{highways}, \textit{vineyard}, \textit{tennis court} and \textit{swimming pool} using a neural network. In the next step, these predictions are georeferenced and the actual changes are determined by comparing the old vector data from the cadastral survey to the predictions. Finally, the changes are sent back to QGIS and visualized there accordingly to their type (changed, deleted, added). Due to the fact, the network has been trained using satellite imagery from Microsoft Bing and vector data from OpenStreetMap, only about 80\% - 90\% of the predictions are correct. This is a result from the fact, that the satellite imagery has to go through a photogrammetric post processing and this can lead to displacements regarding the OpenStreetMap vector data.